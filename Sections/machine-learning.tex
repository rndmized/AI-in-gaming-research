\section{Machine Learning}
Section text here.
\vspace{2mm}

\subsection{Defining Behaviour with AI}
Most games nowadays, either single or multi-player, have NPCs. Whether they are enemies, allies, unnamed or story characters it is important to define a behaviour pattern for therm to perform. However, players tend to reject erratic or inconsistent behaviour from agents, breaking game immersion. To create realistic behaviours that act in a more "human-like" manner is the aim of many AI researchers in the video game industry. In some cases, the goal of the agent is not to perform an action as perfect as a computer would, being the most efficient, but rather try to simulate imperfection. As well, in order for NPC to act as other players is necessary to emulate the lack of information a real player might have such as not being able to detect an enemy that is out of sight but act alert to the possibility of him coming. 

\vspace{2mm}
Chek Tien Tan and Ho-lun Cheng[11] proposed an agent personality representation model to provide an adaptable agent that can perform across a variety of games of different genres exhibiting a plausible adapted behaviour. The Tactical Agent Personality (TAP) is a framework model of progressive learning, to test its capabilities and evaluate the adaptability of the model three scenarios were created. In each scenario, TAP had a series of predefined behaviour with randomly assigned weights to start with and after every iteration a process starter to assimilate new weights (greedy search) and to assign a random value to seek new paths.

\vspace{2mm}
In the first scenario the model was tested against a First Person Shooter (FPS) environment. Three sets of experiments were performed. In the first set the AI was determined at random. In the second set the AI adapted its behaviour based on player performance. In the third set the AI adapted its behaviour base on the NPCs performance. The output of 500 tests per set demonstrated that the highest level of adaptability occurred when AI adaptation was based on the NPCs behaviour rather than player performance.

\vspace{2mm}
In the second scenario the game was tested against a Real-Time Strategy (RTS) environment. In addition to TAP, another layer of decision making was added, the Strategic Agent Personality (SAP). While TAP defined NPC behaviour at an individual level (short-term decisions), SAP determined their actions as a group (long-term decisions).

\subsection{Using Machine Learning to Increase Performance}
Subsection text here.
\vspace{2mm}