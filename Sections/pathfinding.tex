\section{Pathfinding}

Pathfinding is a technique or a process carried by a software running in a computer from which the shortest path or route between two points is extrapolated. Pathfinding is not only important in video games, but also in other areas such as delivery, transport services and intelligent storage among others. Pathfinding algorithms are typically, although not solely, heavily based on the Dijkstra's algorithm[5] which implies what follows: Let the node starting be X. Let the distance of node Y be the distance from X to Y. Dijkstra's algorithm will assign some initial distance values and will try iteratively to improve them step by step.

\vspace{2mm}
However there are other complementary approaches using different techniques that seek to improve such algorithms.

\subsection{Techniques for Pathfinding}
Pathfinding is usually a two-step process: first, a graph generation algorithm followed by a Pathfinding algorithm to calculate the best path. Zeyad Adb Algfoor et al[0] published a brief but thorough compendium of the most common techniques developed during the last 10 years.


\subsubsection{Grids}
In Pathfinding grids are represented as graphs, a list of vertices or points connected by edges to represent the map, the navigation performance of which is based on its attributes. The most popular grid approaches are: regular grids and irregular grids.

\vspace{2mm}
Regular grids use triangles, squares and hexagons to describe tessellations of a given terrain. The most prominent of which are the following:

\begin{itemize}
	\item 2D hexagonal Grid
	\item 2D Square Grid
	\item 2D Triangular Grid
	\item 3D Cubic Grid
\end{itemize}

\vspace{2mm}
Whereas irregular grids use different techniques to define terrain topology such as:

\begin{itemize}
	\item Visibility Graphs
	\item Mesh Navigation
	\item Waypoints
\end{itemize}

\vspace{2mm}
Further information on the actual implementation of these pathfinding techniques can be found in Zeyad et al[9]'s paper.


\subsection{Using Potential Fields}

In 1985, Ossama Khatib discussed a new concept in regards a real-time obstacle avoidance for manipulators and mobile Robots. He named it Artificial Potential Fields[6]. In 2008, Johan Hagelback et al.[10] used it to create a Multi-agent Potential Field-based Bot for Real-Time Strategy Games(RTS) in which objects in the map where assigned charges (like magnetic chargers positive or negative) and such charges were used as forces to attract the agents to specific destination or to repel such agents to avoid collisions or to delimit the terrain.

\vspace{2mm}
They proposed two-different scenarios using Open real-time strategy (ORTS) engine to deploy the bot. In the first scenario the task of the bot was to recollect as many resource as possible in a limited amount of time. To do so Johan Hagelback et al. had 20 agents (workers) using a finite state machine (FSM) to define their behaviour and assigned the charges to the agents, to the resource gathering point, the base were resources had to be delivered and "sheep" moving around the map to provide moving obstacles. Even though the results shown that the bot was prone to disconnect from the game in some occasions, it beat bots developed in previous years improving the average of resources gathered.

\vspace{2mm}
The second scenario, using ORTS, was a tank game where two players had a number of tanks and bases and the goal of the game was to destroy both enemy bases and tanks. After outlining the rules for the game agents, they assigned the charges accordingly and programmed the bot to avoid situating tanks in the middle of enemy clusters. The results of such scenario, tested against other bots, were of a 98\% of victory rate over their opponents.


\subsection{Aesthetics in Pathfinding}

While most of the algorithms focuses lies on obtaining the best suitable path, such path does not always meet the expectations by using an unrealistic behaviour that differs greatly of what a humans would perform. In order to find more "human-like" behaviour in Pathfinding some authors such as Coleman, R.[7] introduce the idea of Aesthetics in it.

\vspace{2mm} 
The experiment's approach was to use Mandelbrot's Fractal Dimension Analysis[8] to determine the aesthetics level of certain paths, an A* Pathfinding algorithm was used to create the path and then tweaked to reward paths that were obstacle-prone to simulate a behaviour of "stealthyness". By doing so the levels of aesthetics were higher than in those were paths were determined as best suited by the algorithm.   