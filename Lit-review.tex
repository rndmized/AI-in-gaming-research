
%% bare_jrnl_compsoc.tex
%% V1.4b
%% 2015/08/26
%% by Michael Shell
%% See:
%% http://www.michaelshell.org/
%% for current contact information.
%%
%% This is a skeleton file demonstrating the use of IEEEtran.cls
%% (requires IEEEtran.cls version 1.8b or later) with an IEEE
%% Computer Society journal paper.
%%
%% Support sites:
%% http://www.michaelshell.org/tex/ieeetran/
%% http://www.ctan.org/pkg/ieeetran
%% and
%% http://www.ieee.org/

%%*************************************************************************
%% Legal Notice:
%% This code is offered as-is without any warranty either expressed or
%% implied; without even the implied warranty of MERCHANTABILITY or
%% FITNESS FOR A PARTICULAR PURPOSE! 
%% User assumes all risk.
%% In no event shall the IEEE or any contributor to this code be liable for
%% any damages or losses, including, but not limited to, incidental,
%% consequential, or any other damages, resulting from the use or misuse
%% of any information contained here.
%%
%% All comments are the opinions of their respective authors and are not
%% necessarily endorsed by the IEEE.
%%
%% This work is distributed under the LaTeX Project Public License (LPPL)
%% ( http://www.latex-project.org/ ) version 1.3, and may be freely used,
%% distributed and modified. A copy of the LPPL, version 1.3, is included
%% in the base LaTeX documentation of all distributions of LaTeX released
%% 2003/12/01 or later.
%% Retain all contribution notices and credits.
%% ** Modified files should be clearly indicated as such, including  **
%% ** renaming them and changing author support contact information. **
%%*************************************************************************


% *** Authors should verify (and, if needed, correct) their LaTeX system  ***
% *** with the testflow diagnostic prior to trusting their LaTeX platform ***
% *** with production work. The IEEE's font choices and paper sizes can   ***
% *** trigger bugs that do not appear when using other class files.       ***                          ***
% The testflow support page is at:
% http://www.michaelshell.org/tex/testflow/


\documentclass[10pt,journal,compsoc]{IEEEtran}
%
% If IEEEtran.cls has not been installed into the LaTeX system files,
% manually specify the path to it like:
% \documentclass[10pt,journal,compsoc]{../sty/IEEEtran}





% Some very useful LaTeX packages include:
% (uncomment the ones you want to load)


% *** MISC UTILITY PACKAGES ***
%
%\usepackage{ifpdf}
% Heiko Oberdiek's ifpdf.sty is very useful if you need conditional
% compilation based on whether the output is pdf or dvi.
% usage:
% \ifpdf
%   % pdf code
% \else
%   % dvi code
% \fi
% The latest version of ifpdf.sty can be obtained from:
% http://www.ctan.org/pkg/ifpdf
% Also, note that IEEEtran.cls V1.7 and later provides a builtin
% \ifCLASSINFOpdf conditional that works the same way.
% When switching from latex to pdflatex and vice-versa, the compiler may
% have to be run twice to clear warning/error messages.






% *** CITATION PACKAGES ***
%
\ifCLASSOPTIONcompsoc
  % IEEE Computer Society needs nocompress option
  % requires cite.sty v4.0 or later (November 2003)
  \usepackage[nocompress]{cite}
\else
  % normal IEEE
  \usepackage{cite}
\fi
% cite.sty was written by Donald Arseneau
% V1.6 and later of IEEEtran pre-defines the format of the cite.sty package
% \cite{} output to follow that of the IEEE. Loading the cite package will
% result in citation numbers being automatically sorted and properly
% "compressed/ranged". e.g., [1], [9], [2], [7], [5], [6] without using
% cite.sty will become [1], [2], [5]--[7], [9] using cite.sty. cite.sty's
% \cite will automatically add leading space, if needed. Use cite.sty's
% noadjust option (cite.sty V3.8 and later) if you want to turn this off
% such as if a citation ever needs to be enclosed in parenthesis.
% cite.sty is already installed on most LaTeX systems. Be sure and use
% version 5.0 (2009-03-20) and later if using hyperref.sty.
% The latest version can be obtained at:
% http://www.ctan.org/pkg/cite
% The documentation is contained in the cite.sty file itself.
%
% Note that some packages require special options to format as the Computer
% Society requires. In particular, Computer Society  papers do not use
% compressed citation ranges as is done in typical IEEE papers
% (e.g., [1]-[4]). Instead, they list every citation separately in order
% (e.g., [1], [2], [3], [4]). To get the latter we need to load the cite
% package with the nocompress option which is supported by cite.sty v4.0
% and later. Note also the use of a CLASSOPTION conditional provided by
% IEEEtran.cls V1.7 and later.





% *** GRAPHICS RELATED PACKAGES ***
%
\ifCLASSINFOpdf
  % \usepackage[pdftex]{graphicx}
  % declare the path(s) where your graphic files are
  % \graphicspath{{../pdf/}{../jpeg/}}
  % and their extensions so you won't have to specify these with
  % every instance of \includegraphics
  % \DeclareGraphicsExtensions{.pdf,.jpeg,.png}
\else
  % or other class option (dvipsone, dvipdf, if not using dvips). graphicx
  % will default to the driver specified in the system graphics.cfg if no
  % driver is specified.
  % \usepackage[dvips]{graphicx}
  % declare the path(s) where your graphic files are
  % \graphicspath{{../eps/}}
  % and their extensions so you won't have to specify these with
  % every instance of \includegraphics
  % \DeclareGraphicsExtensions{.eps}
\fi
% graphicx was written by David Carlisle and Sebastian Rahtz. It is
% required if you want graphics, photos, etc. graphicx.sty is already
% installed on most LaTeX systems. The latest version and documentation
% can be obtained at: 
% http://www.ctan.org/pkg/graphicx
% Another good source of documentation is "Using Imported Graphics in
% LaTeX2e" by Keith Reckdahl which can be found at:
% http://www.ctan.org/pkg/epslatex
%
% latex, and pdflatex in dvi mode, support graphics in encapsulated
% postscript (.eps) format. pdflatex in pdf mode supports graphics
% in .pdf, .jpeg, .png and .mps (metapost) formats. Users should ensure
% that all non-photo figures use a vector format (.eps, .pdf, .mps) and
% not a bitmapped formats (.jpeg, .png). The IEEE frowns on bitmapped formats
% which can result in "jaggedy"/blurry rendering of lines and letters as
% well as large increases in file sizes.
%
% You can find documentation about the pdfTeX application at:
% http://www.tug.org/applications/pdftex






% *** MATH PACKAGES ***
%
%\usepackage{amsmath}
% A popular package from the American Mathematical Society that provides
% many useful and powerful commands for dealing with mathematics.
%
% Note that the amsmath package sets \interdisplaylinepenalty to 10000
% thus preventing page breaks from occurring within multiline equations. Use:
%\interdisplaylinepenalty=2500
% after loading amsmath to restore such page breaks as IEEEtran.cls normally
% does. amsmath.sty is already installed on most LaTeX systems. The latest
% version and documentation can be obtained at:
% http://www.ctan.org/pkg/amsmath





% *** SPECIALIZED LIST PACKAGES ***
%
%\usepackage{algorithmic}
% algorithmic.sty was written by Peter Williams and Rogerio Brito.
% This package provides an algorithmic environment fo describing algorithms.
% You can use the algorithmic environment in-text or within a figure
% environment to provide for a floating algorithm. Do NOT use the algorithm
% floating environment provided by algorithm.sty (by the same authors) or
% algorithm2e.sty (by Christophe Fiorio) as the IEEE does not use dedicated
% algorithm float types and packages that provide these will not provide
% correct IEEE style captions. The latest version and documentation of
% algorithmic.sty can be obtained at:
% http://www.ctan.org/pkg/algorithms
% Also of interest may be the (relatively newer and more customizable)
% algorithmicx.sty package by Szasz Janos:
% http://www.ctan.org/pkg/algorithmicx




% *** ALIGNMENT PACKAGES ***
%
%\usepackage{array}
% Frank Mittelbach's and David Carlisle's array.sty patches and improves
% the standard LaTeX2e array and tabular environments to provide better
% appearance and additional user controls. As the default LaTeX2e table
% generation code is lacking to the point of almost being broken with
% respect to the quality of the end results, all users are strongly
% advised to use an enhanced (at the very least that provided by array.sty)
% set of table tools. array.sty is already installed on most systems. The
% latest version and documentation can be obtained at:
% http://www.ctan.org/pkg/array


% IEEEtran contains the IEEEeqnarray family of commands that can be used to
% generate multiline equations as well as matrices, tables, etc., of high
% quality.




% *** SUBFIGURE PACKAGES ***
%\ifCLASSOPTIONcompsoc
%  \usepackage[caption=false,font=footnotesize,labelfont=sf,textfont=sf]{subfig}
%\else
%  \usepackage[caption=false,font=footnotesize]{subfig}
%\fi
% subfig.sty, written by Steven Douglas Cochran, is the modern replacement
% for subfigure.sty, the latter of which is no longer maintained and is
% incompatible with some LaTeX packages including fixltx2e. However,
% subfig.sty requires and automatically loads Axel Sommerfeldt's caption.sty
% which will override IEEEtran.cls' handling of captions and this will result
% in non-IEEE style figure/table captions. To prevent this problem, be sure
% and invoke subfig.sty's "caption=false" package option (available since
% subfig.sty version 1.3, 2005/06/28) as this is will preserve IEEEtran.cls
% handling of captions.
% Note that the Computer Society format requires a sans serif font rather
% than the serif font used in traditional IEEE formatting and thus the need
% to invoke different subfig.sty package options depending on whether
% compsoc mode has been enabled.
%
% The latest version and documentation of subfig.sty can be obtained at:
% http://www.ctan.org/pkg/subfig




% *** FLOAT PACKAGES ***
%
%\usepackage{fixltx2e}
% fixltx2e, the successor to the earlier fix2col.sty, was written by
% Frank Mittelbach and David Carlisle. This package corrects a few problems
% in the LaTeX2e kernel, the most notable of which is that in current
% LaTeX2e releases, the ordering of single and double column floats is not
% guaranteed to be preserved. Thus, an unpatched LaTeX2e can allow a
% single column figure to be placed prior to an earlier double column
% figure.
% Be aware that LaTeX2e kernels dated 2015 and later have fixltx2e.sty's
% corrections already built into the system in which case a warning will
% be issued if an attempt is made to load fixltx2e.sty as it is no longer
% needed.
% The latest version and documentation can be found at:
% http://www.ctan.org/pkg/fixltx2e


%\usepackage{stfloats}
% stfloats.sty was written by Sigitas Tolusis. This package gives LaTeX2e
% the ability to do double column floats at the bottom of the page as well
% as the top. (e.g., "\begin{figure*}[!b]" is not normally possible in
% LaTeX2e). It also provides a command:
%\fnbelowfloat
% to enable the placement of footnotes below bottom floats (the standard
% LaTeX2e kernel puts them above bottom floats). This is an invasive package
% which rewrites many portions of the LaTeX2e float routines. It may not work
% with other packages that modify the LaTeX2e float routines. The latest
% version and documentation can be obtained at:
% http://www.ctan.org/pkg/stfloats
% Do not use the stfloats baselinefloat ability as the IEEE does not allow
% \baselineskip to stretch. Authors submitting work to the IEEE should note
% that the IEEE rarely uses double column equations and that authors should try
% to avoid such use. Do not be tempted to use the cuted.sty or midfloat.sty
% packages (also by Sigitas Tolusis) as the IEEE does not format its papers in
% such ways.
% Do not attempt to use stfloats with fixltx2e as they are incompatible.
% Instead, use Morten Hogholm'a dblfloatfix which combines the features
% of both fixltx2e and stfloats:
%
% \usepackage{dblfloatfix}
% The latest version can be found at:
% http://www.ctan.org/pkg/dblfloatfix




%\ifCLASSOPTIONcaptionsoff
%  \usepackage[nomarkers]{endfloat}
% \let\MYoriglatexcaption\caption
% \renewcommand{\caption}[2][\relax]{\MYoriglatexcaption[#2]{#2}}
%\fi
% endfloat.sty was written by James Darrell McCauley, Jeff Goldberg and 
% Axel Sommerfeldt. This package may be useful when used in conjunction with 
% IEEEtran.cls'  captionsoff option. Some IEEE journals/societies require that
% submissions have lists of figures/tables at the end of the paper and that
% figures/tables without any captions are placed on a page by themselves at
% the end of the document. If needed, the draftcls IEEEtran class option or
% \CLASSINPUTbaselinestretch interface can be used to increase the line
% spacing as well. Be sure and use the nomarkers option of endfloat to
% prevent endfloat from "marking" where the figures would have been placed
% in the text. The two hack lines of code above are a slight modification of
% that suggested by in the endfloat docs (section 8.4.1) to ensure that
% the full captions always appear in the list of figures/tables - even if
% the user used the short optional argument of \caption[]{}.
% IEEE papers do not typically make use of \caption[]'s optional argument,
% so this should not be an issue. A similar trick can be used to disable
% captions of packages such as subfig.sty that lack options to turn off
% the subcaptions:
% For subfig.sty:
% \let\MYorigsubfloat\subfloat
% \renewcommand{\subfloat}[2][\relax]{\MYorigsubfloat[]{#2}}
% However, the above trick will not work if both optional arguments of
% the \subfloat command are used. Furthermore, there needs to be a
% description of each subfigure *somewhere* and endfloat does not add
% subfigure captions to its list of figures. Thus, the best approach is to
% avoid the use of subfigure captions (many IEEE journals avoid them anyway)
% and instead reference/explain all the subfigures within the main caption.
% The latest version of endfloat.sty and its documentation can obtained at:
% http://www.ctan.org/pkg/endfloat
%
% The IEEEtran \ifCLASSOPTIONcaptionsoff conditional can also be used
% later in the document, say, to conditionally put the References on a 
% page by themselves.




% *** PDF, URL AND HYPERLINK PACKAGES ***
%
%\usepackage{url}
% url.sty was written by Donald Arseneau. It provides better support for
% handling and breaking URLs. url.sty is already installed on most LaTeX
% systems. The latest version and documentation can be obtained at:
% http://www.ctan.org/pkg/url
% Basically, \url{my_url_here}.





% *** Do not adjust lengths that control margins, column widths, etc. ***
% *** Do not use packages that alter fonts (such as pslatex).         ***
% There should be no need to do such things with IEEEtran.cls V1.6 and later.
% (Unless specifically asked to do so by the journal or conference you plan
% to submit to, of course. )


% correct bad hyphenation here
\hyphenation{op-tical net-works semi-conduc-tor}


\begin{document}
%
% paper title
% Titles are generally capitalized except for words such as a, an, and, as,
% at, but, by, for, in, nor, of, on, or, the, to and up, which are usually
% not capitalized unless they are the first or last word of the title.
% Linebreaks \\ can be used within to get better formatting as desired.
% Do not put math or special symbols in the title.
\title{Practical Applications of AI and Machine learning in different areas within the Game Development Industry and multiple implementation approaches.}
%
%
% author names and IEEE memberships
% note positions of commas and nonbreaking spaces ( ~ ) LaTeX will not break
% a structure at a ~ so this keeps an author's name from being broken across
% two lines.
% use \thanks{} to gain access to the first footnote area
% a separate \thanks must be used for each paragraph as LaTeX2e's \thanks
% was not built to handle multiple paragraphs
%
%
%\IEEEcompsocitemizethanks is a special \thanks that produces the bulleted
% lists the Computer Society journals use for "first footnote" author
% affiliations. Use \IEEEcompsocthanksitem which works much like \item
% for each affiliation group. When not in compsoc mode,
% \IEEEcompsocitemizethanks becomes like \thanks and
% \IEEEcompsocthanksitem becomes a line break with idention. This
% facilitates dual compilation, although admittedly the differences in the
% desired content of \author between the different types of papers makes a
% one-size-fits-all approach a daunting prospect. For instance, compsoc 
% journal papers have the author affiliations above the "Manuscript
% received ..."  text while in non-compsoc journals this is reversed. Sigh.

\author{Albert~Rando,~RESEARCH METHODS IN COMPUTING AND IT~-~GMIT, \\ G00330058
	\\Professor - Dr. Martin Kenirons}% <-this % stops a space


% note the % following the last \IEEEmembership and also \thanks - 
% these prevent an unwanted space from occurring between the last author name
% and the end of the author line. i.e., if you had this:
% 
% \author{....lastname \thanks{...} \thanks{...} }
%                     ^------------^------------^----Do not want these spaces!
%
% a space would be appended to the last name and could cause every name on that
% line to be shifted left slightly. This is one of those "LaTeX things". For
% instance, "\textbf{A} \textbf{B}" will typeset as "A B" not "AB". To get
% "AB" then you have to do: "\textbf{A}\textbf{B}"
% \thanks is no different in this regard, so shield the last } of each \thanks
% that ends a line with a % and do not let a space in before the next \thanks.
% Spaces after \IEEEmembership other than the last one are OK (and needed) as
% you are supposed to have spaces between the names. For what it is worth,
% this is a minor point as most people would not even notice if the said evil
% space somehow managed to creep in.



% The paper headers
\markboth{International Journal of Computer Games Technology, November~2017}%
{Shell \MakeLowercase{\textit{et al.}}: Bare Demo of IEEEtran.cls for Computer Society Journals}
% The only time the second header will appear is for the odd numbered pages
% after the title page when using the twoside option.
% 
% *** Note that you probably will NOT want to include the author's ***
% *** name in the headers of peer review papers.                   ***
% You can use \ifCLASSOPTIONpeerreview for conditional compilation here if
% you desire.



% The publisher's ID mark at the bottom of the page is less important with
% Computer Society journal papers as those publications place the marks
% outside of the main text columns and, therefore, unlike regular IEEE
% journals, the available text space is not reduced by their presence.
% If you want to put a publisher's ID mark on the page you can do it like
% this:
%\IEEEpubid{0000--0000/00\$00.00~\copyright~2015 IEEE}
% or like this to get the Computer Society new two part style.
%\IEEEpubid{\makebox[\columnwidth]{\hfill 0000--0000/00/\$00.00~\copyright~2015 IEEE}%
%\hspace{\columnsep}\makebox[\columnwidth]{Published by the IEEE Computer Society\hfill}}
% Remember, if you use this you must call \IEEEpubidadjcol in the second
% column for its text to clear the IEEEpubid mark (Computer Society jorunal
% papers don't need this extra clearance.)



% use for special paper notices
%\IEEEspecialpapernotice{(Invited Paper)}



% for Computer Society papers, we must declare the abstract and index terms
% PRIOR to the title within the \IEEEtitleabstractindextext IEEEtran
% command as these need to go into the title area created by \maketitle.
% As a general rule, do not put math, special symbols or citations
% in the abstract or keywords.
\IEEEtitleabstractindextext{%
\begin{abstract}
This paper draws a rough sketch of what the gaming industry represents in the current world and how AI and Machine Learning became almost core to it. It explains some areas in video games where AI is implemented and the different approaches used in order to maximize "human like" behaviour when required and how AI can be used to improve performance in pathfinding for the game agents.
\end{abstract}

% Note that keywords are not normally used for peerreview papers.
\begin{IEEEkeywords}
AI, Machine Learning, Gaming, Game Design, Pathfinding, Agents, RTS, AI Behaviour.
\end{IEEEkeywords}}


% make the title area
\maketitle


% To allow for easy dual compilation without having to reenter the
% abstract/keywords data, the \IEEEtitleabstractindextext text will
% not be used in maketitle, but will appear (i.e., to be "transported")
% here as \IEEEdisplaynontitleabstractindextext when the compsoc 
% or transmag modes are not selected <OR> if conference mode is selected 
% - because all conference papers position the abstract like regular
% papers do.
\IEEEdisplaynontitleabstractindextext
% \IEEEdisplaynontitleabstractindextext has no effect when using
% compsoc or transmag under a non-conference mode.



% For peer review papers, you can put extra information on the cover
% page as needed:
% \ifCLASSOPTIONpeerreview
% \begin{center} \bfseries EDICS Category: 3-BBND \end{center}
% \fi
%
% For peerreview papers, this IEEEtran command inserts a page break and
% creates the second title. It will be ignored for other modes.
\IEEEpeerreviewmaketitle



\IEEEraisesectionheading{\section{Introduction}\label{sec:introduction}}
% Computer Society journal (but not conference!) papers do something unusual
% with the very first section heading (almost always called "Introduction").
% They place it ABOVE the main text! IEEEtran.cls does not automatically do
% this for you, but you can achieve this effect with the provided
% \IEEEraisesectionheading{} command. Note the need to keep any \label that
% is to refer to the section immediately after \section in the above as
% \IEEEraisesectionheading puts \section within a raised box.




% The very first letter is a 2 line initial drop letter followed
% by the rest of the first word in caps (small caps for compsoc).
% 
% form to use if the first word consists of a single letter:
% \IEEEPARstart{A}{demo} file is ....
% 
% form to use if you need the single drop letter followed by
% normal text (unknown if ever used by the IEEE):
% \IEEEPARstart{A}{}demo file is ....
% 
% Some journals put the first two words in caps:
% \IEEEPARstart{T}{his demo} file is ....
% 
% Here we have the typical use of a "T" for an initial drop letter
% and "HIS" in caps to complete the first word.
\IEEEPARstart{W}{e} have come a long way since first video games were created back in the 1950's. The first of which was 'Tennis for Two', introduced on October 18, 1958, at Brookhaven National Laboratory, An electronic tennis game that would become the father of modern video games[1]. But what is essentially a video game? As defined by [2] video game [noun] "A game played by electronically manipulating images produced by a computer program on a monitor or other display". Born in laboratories the first video games were developed by scientists putting the computational frames on Universities to the test. 

\vspace{2mm}
On the other hand Artificial Intelligence as a concept was born earlier culturally in the form of robots on films and comic books. Alan Turing a British Mathematician started to tread the path by suggesting that just like human process information to make decisions, machines could equally do it. However, it was not until the 1950's that computational capabilities were powerful enough, not only to perform operations but also store to commands, to do some actual practical research. AI as such was conceived as such in a famous conference presented at the Dartmouth Summer Research Project on Artificial Intelligence (DSRPAI) hosted by John McCarthy and Marvin Minsky in 1956 [3].

\vspace{2mm}
As technology progressed very much in accordance to Moore's law, the increase in computational power allowed both fields, AI and Video Games to develop even further from the expected. As video games have grown both in scope and depth, AI has become an entwined part of it. From bots capable of defeating humans in games such as Chess or Go, to better gaming experiences and more realistic behaviours through NPC's (Non-playable Characters), AI provides with more enjoyable challenges an ever growing sector. According to a newzoo article[4] about the Gaming Industry's revenue in 2017, it is expected that 2.2 billion gamers worldwide will generate \$108.9 billion in this year's exercise, which represents a significant increase from previous years. 

\vspace{2mm}
In this paper, we pretend to introduce some of the most common uses for AI that predominate in video games, and provide a definition as well as to explain some approaches to implement them. We will also provide an insight of how machine learning has improved both capabilities and performance of some of the proposed models and algorithms and how can they be tuned to increase outputs.

\subsection{Pathfinding}


Pathfinding is a technique or a process carried by a software running in a computer from which the shortest path or route between two points is extrapolated. Pathfinding is, not only important in video games, but also in other areas such as delivery, transport services and intelligent storage among others. Pathfinding algorithms are usually heavily based on the Dijkstra's algorithm[5] which implies what follows: Let the node starting be X. Let the distance of node Y be the distance from X to Y. Dijkstra's algorithm will assign some initial distance values and will try iteratively to improve them step by step.

\vspace{2mm}
However there are other approaches using different techniques.



\subsubsection{Using Potential Fields}
In 1985, Ossama Khatib discussed a new concept in regards a real-time obstacle avoidance for manipulators and mobile Robots. He named it Artificial Potential Fields[6]. In 2008, Johan Hagelback et al.[7] used it to create a Multi-agent Potential Field-based Bot for Real-Time Strategy Games(RTS) in which objects in the map where assigned charges (like magnetic chargers positive or negative) and such charges were used as forces to attract the agents to specific destination or to repel such agents to avoid collisions or to delimit the terrain.

\vspace{2mm}
They proposed two-different scenarios using Open real-time strategy (ORTS) engine to deploy the bot. In the first scenario the task of the bot was to recollect as many resource as possible in a limited amount of time. To do so Johan Hagelback et al. had 20 agents (workers) using a finite state machine (FSM) to define their behaviour and assigned the charges to the agents, to the resource gathering point, the base were resources had to be delivered and "sheep" moving around the map to provide moving obstacles. Even though the results shown that the bot was prone to disconnect from the game in some occasions, it beat bots developed in previous years improving the average of resources gathered.

\vspace{2mm}
The second scenario, using ORTS, was a tank game where two players had a number of tanks and bases and the goal of the game was to destroy both enemy bases and tanks. After outlining the rules for the game agents, they assigned the charges accordingly and programmed the bot to avoid situating tanks in the middle of enemy clusters. The results of such scenario, tested against other bots, were of a 98\% of victory rate over their opponents.
\section{Machine Learning}
Even though Machine Learning has been a field of stud in computer science for the last 30 years, it was not until very recently that game developers started to implement it in their games. The lack of enthusiasm for adaptable behaviour in games typically responds to the fact that it is not really that important for a game to "learn". However, more and more developers release their games as services rather than products. Replayablity has become a huge factor, if using the same strategy over and over achieves always victory it gets tedious, more a chore than a challenge. A game that can learn adapt to its players has a higher chance to provide its users with more enjoyable situations.

\subsection{Defining Behaviour with AI}
Most games nowadays, either single or multi-player, have NPCs. Whether they are enemies, allies, unnamed or story characters it is important to define a behaviour pattern for therm to perform. However, players tend to reject erratic or inconsistent behaviour from agents, breaking game immersion. To create realistic behaviours that act in a more "human-like" manner is the aim of many AI researchers in the video game industry. In some cases, the goal of the agent is not to perform an action as perfect as a computer would, being the most efficient, but rather try to simulate imperfection. As well, in order for NPC to act as other players is necessary to emulate the lack of information a real player might have such as not being able to detect an enemy that is out of sight but act alert to the possibility of him coming. With that in mind Marvin T. Chan et al [12] designed an AI that emulated driving like a human would do through a race track providing a more challenging experience for the player. As stated in the paper "Although the player’s goal within the game is to win the race against the game-controlled car, the AI techniques adopted in the game are primarily designed to give the player an enjoyable time racing his or her car. In other words, the objective of winning by either side is not given the highest priority."
 


\vspace{2mm}
Defining the behaviour of the AI requires the expertise to be able to analyse and describe accurately such behaviour and script it accordingly. Chek Tien Tan and Ho-lun Cheng[11] proposed an agent personality representation model to provide an adaptable agent that can perform across a variety of games of different genres exhibiting a plausible adapted behaviour. The Tactical Agent Personality (TAP) is a framework model of progressive learning, to test its capabilities and evaluate the adaptability of the model three scenarios were created. In each scenario, TAP had a series of predefined behaviour with randomly assigned weights to start with and after every iteration a process starter to assimilate new weights (greedy search) and to assign a random value to seek new paths.

\vspace{2mm}
In the first scenario the model was tested against a First Person Shooter (FPS) environment. Three sets of experiments were performed. In the first set the AI was determined at random. In the second set the AI adapted its behaviour based on player performance. In the third set the AI adapted its behaviour base on the NPCs performance. The output of 500 tests per set demonstrated that the highest level of adaptability occurred when AI adaptation was based on the NPCs behaviour rather than player performance.

\vspace{2mm}
In the second scenario the model was tested against a Real-Time Strategy (RTS) environment. In addition to TAP, another layer of decision making was added, the Strategic Agent Personality (SAP). While TAP defined NPC behaviour at an individual level (short-term decisions), SAP determined their actions as a group (long-term decisions).

\vspace{2mm}
In the third scenario the model was tested against a Role-Playing Game (RPG) environment. In this test, the model was modified to switch the weight from the nodes or actions to the edges representing the temporary transitions between actions, Temporary Tactical Agent Personality (TTAP). The game consisted in two groups of characters 1 player and 5 NPCs. The first group implemented TAP while the second implemented TTAP. After running the experiment 500 times, the TTAP demonstrated a quicker adaptation but at the 500 the difference between the two models grow shorter and draws became more frequent.

\vspace{2mm}
TAP presents some interesting points such as high versatility and adaptation that reduces the need of specific scripting for actions escaping from more traditional approaches like Finite State Machines. TAP and SAP present a good performance and the scalability has low impact, however TTAP does not represent a better model over time and presents scalability issues as calculating the weight on the edges increments the overhead by a significant amount.

\subsection{Perfect versus Imperfect Knowledge}
As previously mentioned, in modern player-bot engagement-based video
games, bot responses and behaviour are designed to be
as realistic as possible. Is because of that, bots are designed to detect players just within its visibility area and not further than that. That is called Imperfect Knowledge. In games such as checkers or chess, both player and computer have complete visibility of the area (board) and all information about what is in it, thus we call this Perfect Knowledge. To provide the player with a sense of fairness it is necessary to program the bot to behave as it has imperfect knowledge. To do so several approaches make use of the principle of "neighbourhood"

\vspace{2mm}
In their article, Peter K. K. Loh and Edmond C. Prakash, evaluate the performance of the existing Moving Target Search Algorithms through simulation:
\begin{itemize}
	\item Basic Moving Target Search (BMTS)
	\item Weighted Moving Target Search (WMTS)
	\item Commitment and Deliberation Moving Target Search (CDMTS)
\end{itemize}
\vspace{2mm}
After simulation the results are compared against the Abstraction Moving Target Search Algorithm (AMTS) design, proposed in their paper. When compared, AMTS outperforms the three MST algorithms, with higher exploration moves, but the lowest MTS move. However further investigation is required as scalability might become an issue. Even though it has downsides, the algorithm presents a interesting approach in the matter.

\vspace{2mm}
Another approach includes Machine Learning to learn and generate strategies to increase performance while maintaining imperfect knowledge. Beaulac and Larribe [13] proposed a model based on Hidden Markov Models to narrow down the possible locations of a hidden mobile agent and pair it with machine learning to create a heat map where "hotter" areas are more likely to be traversed by such agent and design strategies around that. To test the model, the experiments presents a simple game (pirate-themed), where the player has to reach two different points in the map. There are "parrots" that indicate to the AI if the mobile agent is in its vicinity, providing more information but still maintaining the imperfect knowledge premise. The model demonstrated its effectiveness but the model requires further adjustments. 
\section{Conclusion}
Our aim in this research is to discuss some of the approaches to AI implementation in video games and what are the higher impact areas in that regard. We also described in what manned the industry requires AI to be developed to match players expectations. It is still a field in expansion, and even though more than 60 years have elapsed since the creation of the first video game and almost as much since the first game implementing a basic AI system, video games evolved always using cutting edge technologies to deliver better experiences. Machine learning provided video games with a new edge. However due to this very same reason, there are little to no standards or frameworks that unify such technologies. The future of AI in gaming, as approached by Safadi el al.[15], by conceptualizing video games, "it becomes possible
to create solutions for common conceptual problems and use them across multiple video games. Developing solutions for conceptual problems rather than specific video games means that AI design is no longer confined to the scope of
individual game projects and can be more efficiently refined over time".

\vspace{2mm}
A unified framework would serve as a core, around which game developers would be able to build add-ons, decreasing the amount of resources for creating pathfinding models and behaviours from scratch and providing a stronger foundation for AI in gaming.






% An example of a floating figure using the graphicx package.
% Note that \label must occur AFTER (or within) \caption.
% For figures, \caption should occur after the \includegraphics.
% Note that IEEEtran v1.7 and later has special internal code that
% is designed to preserve the operation of \label within \caption
% even when the captionsoff option is in effect. However, because
% of issues like this, it may be the safest practice to put all your
% \label just after \caption rather than within \caption{}.
%
% Reminder: the "draftcls" or "draftclsnofoot", not "draft", class
% option should be used if it is desired that the figures are to be
% displayed while in draft mode.
%
%\begin{figure}[!t]
%\centering
%\includegraphics[width=2.5in]{myfigure}
% where an .eps filename suffix will be assumed under latex, 
% and a .pdf suffix will be assumed for pdflatex; or what has been declared
% via \DeclareGraphicsExtensions.
%\caption{Simulation results for the network.}
%\label{fig_sim}
%\end{figure}

% Note that the IEEE typically puts floats only at the top, even when this
% results in a large percentage of a column being occupied by floats.
% However, the Computer Society has been known to put floats at the bottom.


% An example of a double column floating figure using two subfigures.
% (The subfig.sty package must be loaded for this to work.)
% The subfigure \label commands are set within each subfloat command,
% and the \label for the overall figure must come after \caption.
% \hfil is used as a separator to get equal spacing.
% Watch out that the combined width of all the subfigures on a 
% line do not exceed the text width or a line break will occur.
%
%\begin{figure*}[!t]
%\centering
%\subfloat[Case I]{\includegraphics[width=2.5in]{box}%
%\label{fig_first_case}}
%\hfil
%\subfloat[Case II]{\includegraphics[width=2.5in]{box}%
%\label{fig_second_case}}
%\caption{Simulation results for the network.}
%\label{fig_sim}
%\end{figure*}
%
% Note that often IEEE papers with subfigures do not employ subfigure
% captions (using the optional argument to \subfloat[]), but instead will
% reference/describe all of them (a), (b), etc., within the main caption.
% Be aware that for subfig.sty to generate the (a), (b), etc., subfigure
% labels, the optional argument to \subfloat must be present. If a
% subcaption is not desired, just leave its contents blank,
% e.g., \subfloat[].


% An example of a floating table. Note that, for IEEE style tables, the
% \caption command should come BEFORE the table and, given that table
% captions serve much like titles, are usually capitalized except for words
% such as a, an, and, as, at, but, by, for, in, nor, of, on, or, the, to
% and up, which are usually not capitalized unless they are the first or
% last word of the caption. Table text will default to \footnotesize as
% the IEEE normally uses this smaller font for tables.
% The \label must come after \caption as always.
%
%\begin{table}[!t]
%% increase table row spacing, adjust to taste
%\renewcommand{\arraystretch}{1.3}
% if using array.sty, it might be a good idea to tweak the value of
% \extrarowheight as needed to properly center the text within the cells
%\caption{An Example of a Table}
%\label{table_example}
%\centering
%% Some packages, such as MDW tools, offer better commands for making tables
%% than the plain LaTeX2e tabular which is used here.
%\begin{tabular}{|c||c|}
%\hline
%One & Two\\
%\hline
%Three & Four\\
%\hline
%\end{tabular}
%\end{table}


% Note that the IEEE does not put floats in the very first column
% - or typically anywhere on the first page for that matter. Also,
% in-text middle ("here") positioning is typically not used, but it
% is allowed and encouraged for Computer Society conferences (but
% not Computer Society journals). Most IEEE journals/conferences use
% top floats exclusively. 
% Note that, LaTeX2e, unlike IEEE journals/conferences, places
% footnotes above bottom floats. This can be corrected via the
% \fnbelowfloat command of the stfloats package.


% Can use something like this to put references on a page
% by themselves when using endfloat and the captionsoff option.
\ifCLASSOPTIONcaptionsoff
  \newpage
\fi



% trigger a \newpage just before the given reference
% number - used to balance the columns on the last page
% adjust value as needed - may need to be readjusted if
% the document is modified later
%\IEEEtriggeratref{8}
% The "triggered" command can be changed if desired:
%\IEEEtriggercmd{\enlargethispage{-5in}}

% references section

% can use a bibliography generated by BibTeX as a .bbl file
% BibTeX documentation can be easily obtained at:
% http://mirror.ctan.org/biblio/bibtex/contrib/doc/
% The IEEEtran BibTeX style support page is at:
% http://www.michaelshell.org/tex/ieeetran/bibtex/
%\bibliographystyle{IEEEtran}
% argument is your BibTeX string definitions and bibliography database(s)
%\bibliography{IEEEabrv,../bib/paper}
%
% <OR> manually copy in the resultant .bbl file
% set second argument of \begin to the number of references
% (used to reserve space for the reference number labels box)

\begin{thebibliography}{1}
	

\bibitem{lit-review:tennis}
	Brookheaven~National~Laboratory,~"The First Video Game?",~Brookheaven~National~Laboratory, 1958. [Online] Available: https://www.bnl.gov/about/history/firstvideo.php
\\

\bibitem{lit-review:definition}
	Oxford Dictionaries,~" Video Game", 2017. [Online] Available: "https://en.oxforddictionaries.com/definition/video\_game"
\\

\bibitem{lit-review:darmouth}
	John McCarthy, Marvin L. Minsky, Nathaniel Rochester,and Claude E. Shannon,~"A Proposal for the Dartmouth Summer Research Project on Artificial Intelligence", August 31, 1955.
\\

\bibitem{lit-review:revenue}
	Emma McDonald,~"The Global Games Market Will Reach \$108.9 Billion in 2017 With Mobile Taking 42", Aprl 20, 2017. [Online] Available: https://newzoo.com/insights/articles/the-global-games-market-will-reach-108-9-billion-in-2017-with-mobile-taking-42/
\\

\bibitem{lit-review:dijkstra}
	Dijkstra, E. W. (1959). "A note on two problems in connexion with graphs". Numerische Mathematik. 1: 269–271. doi:10.1007/BF01386390.
\\

\bibitem{lit-review:khatib}
 O. Kathib. "Real-time obstacle avoidance for manipulators and mobile robots". The international Journal of Robotics Research, Vol 5, no.1,pp. 90-98, 1986.
\\

\bibitem{lit-review:aesthetics}
Ron Coleman, “Fractal Analysis of Stealthy Pathfinding Aesthetics,” International Journal of Computer Games Technology, vol. 2009, Article ID 670459, 7 pages, 2009.
\\

\bibitem{lit-review:fractal}
B. Mandelbrot, The Fractal Geometry of Nature, W. H. Freeman, San Francisco, Calif, USA, 1982
\\

\bibitem{lit-review:pathTech}
Zeyad Abd Algfoor, Mohd Shahrizal Sunar, and Hoshang Kolivand, "A Comprehensive Study on Pathfinding Techniques for Robotics and Video Games" International Journal of Computer Games Technology, vol. 2015, Article ID 736138, 11 pages, 2015.
\\

\bibitem{lit-review:potentialFields}
 Johan Hagelbäck and Stefan J. Johansson, “A Multiagent Potential Field-Based Bot for Real-Time Strategy Games,” International Journal of Computer Games Technology, vol. 2009, Article ID 910819, 10 pages, 2009. 
\\

\bibitem{lit-review:tap}
Chek Tien Tan and Ho-lun Cheng, “Tactical Agent Personality,” International Journal of Computer Games Technology, vol. 2011, Article ID 107160, 16 pages, 2011.
\\

\bibitem{lit-review:car}
Marvin T. Chan, Christine W. Chan, and Craig Gelowitz, “Development of a Car Racing Simulator Game Using Artificial Intelligence Techniques,” International Journal of Computer Games Technology, vol. 2015, Article ID 839721, 6 pages, 2015.
\\

\bibitem{lit-review:simulation}
Peter K. K. Loh and Edmond C. Prakash, “Performance Simulations of Moving Target Search Algorithms,” International Journal of Computer Games Technology, vol. 2009, Article ID 745219, 6 pages, 2009. 
\\

\bibitem{lit-review:hmm}
Cédric Beaulac and Fabrice Larribe, “Narrow Artificial Intelligence with Machine Learning for Real-Time Estimation of a Mobile Agent’s Location Using Hidden Markov Models,” International Journal of Computer Games Technology, vol. 2017, Article ID 4939261, 10 pages, 2017. 
\\

\bibitem{lit-review:future}
Firas Safadi, Raphael Fonteneau, and Damien Ernst, “Artificial Intelligence in Video Games: Towards a Unified Framework,” International Journal of Computer Games Technology, vol. 2015, Article ID 271296, 30 pages, 2015. 

\end{thebibliography}


% biography section
% 
% If you have an EPS/PDF photo (graphicx package needed) extra braces are
% needed around the contents of the optional argument to biography to prevent
% the LaTeX parser from getting confused when it sees the complicated
% \includegraphics command within an optional argument. (You could create
% your own custom macro containing the \includegraphics command to make things
% simpler here.)
%\begin{IEEEbiography}[{\includegraphics[width=1in,height=1.25in,clip,keepaspectratio]{mshell}}]{Michael Shell}
% or if you just want to reserve a space for a photo:

\begin{IEEEbiographynophoto}{Albert Rando}
Student in GMIT, currently attending fourth year in BSC in Software Development.
\end{IEEEbiographynophoto}

% insert where needed to balance the two columns on the last page with
% biographies
%\newpage


% You can push biographies down or up by placing
% a \vfill before or after them. The appropriate
% use of \vfill depends on what kind of text is
% on the last page and whether or not the columns
% are being equalized.

\vfill

% Can be used to pull up biographies so that the bottom of the last one
% is flush with the other column.
%\enlargethispage{-5in}

% that's all folks
\end{document}


